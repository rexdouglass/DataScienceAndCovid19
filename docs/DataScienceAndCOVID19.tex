% Options for packages loaded elsewhere
\PassOptionsToPackage{unicode}{hyperref}
\PassOptionsToPackage{hyphens}{url}
%
\documentclass[
]{article}
\usepackage{lmodern}
\usepackage{amssymb,amsmath}
\usepackage{ifxetex,ifluatex}
\ifnum 0\ifxetex 1\fi\ifluatex 1\fi=0 % if pdftex
  \usepackage[T1]{fontenc}
  \usepackage[utf8]{inputenc}
  \usepackage{textcomp} % provide euro and other symbols
\else % if luatex or xetex
  \usepackage{unicode-math}
  \defaultfontfeatures{Scale=MatchLowercase}
  \defaultfontfeatures[\rmfamily]{Ligatures=TeX,Scale=1}
\fi
% Use upquote if available, for straight quotes in verbatim environments
\IfFileExists{upquote.sty}{\usepackage{upquote}}{}
\IfFileExists{microtype.sty}{% use microtype if available
  \usepackage[]{microtype}
  \UseMicrotypeSet[protrusion]{basicmath} % disable protrusion for tt fonts
}{}
\makeatletter
\@ifundefined{KOMAClassName}{% if non-KOMA class
  \IfFileExists{parskip.sty}{%
    \usepackage{parskip}
  }{% else
    \setlength{\parindent}{0pt}
    \setlength{\parskip}{6pt plus 2pt minus 1pt}}
}{% if KOMA class
  \KOMAoptions{parskip=half}}
\makeatother
\usepackage{xcolor}
\IfFileExists{xurl.sty}{\usepackage{xurl}}{} % add URL line breaks if available
\IfFileExists{bookmark.sty}{\usepackage{bookmark}}{\usepackage{hyperref}}
\hypersetup{
  pdftitle={The Data Science of COVID-19 Spread: Some Troubling Current and Future Trends},
  hidelinks,
  pdfcreator={LaTeX via pandoc}}
\urlstyle{same} % disable monospaced font for URLs
\usepackage[margin=1in]{geometry}
\usepackage{graphicx}
\makeatletter
\def\maxwidth{\ifdim\Gin@nat@width>\linewidth\linewidth\else\Gin@nat@width\fi}
\def\maxheight{\ifdim\Gin@nat@height>\textheight\textheight\else\Gin@nat@height\fi}
\makeatother
% Scale images if necessary, so that they will not overflow the page
% margins by default, and it is still possible to overwrite the defaults
% using explicit options in \includegraphics[width, height, ...]{}
\setkeys{Gin}{width=\maxwidth,height=\maxheight,keepaspectratio}
% Set default figure placement to htbp
\makeatletter
\def\fps@figure{htbp}
\makeatother
\setlength{\emergencystretch}{3em} % prevent overfull lines
\providecommand{\tightlist}{%
  \setlength{\itemsep}{0pt}\setlength{\parskip}{0pt}}
\setcounter{secnumdepth}{-\maxdimen} % remove section numbering
\newlength{\cslhangindent}
\setlength{\cslhangindent}{1.5em}
\newenvironment{cslreferences}%
  {\setlength{\parindent}{0pt}%
  \everypar{\setlength{\hangindent}{\cslhangindent}}\ignorespaces}%
  {\par}

\title{The Data Science of COVID-19 Spread: Some Troubling Current and
Future Trends}
\author{true \and true \and true}
\date{19 August, 2020}

\begin{document}
\maketitle

Working draft. Comments welcome! Follow the
\href{https://twitter.com/RexDouglass/status/1295391498477826053}{Twitter
Discussion}.

\hypertarget{introduction}{%
\section{Introduction}\label{introduction}}

The SARS-COV-2 global pandemic has exposed weaknesses throughout our
institutions, and the sciences are no exception. Given the deluge of
official statistics and 300+ new COVID-19 working papers posted each
day\footnote{``COVID-19 Primer.'' Accessed August 17, 2020.
  \url{https://covid19primer.com/}.}, it is imperative for both
consumers and producers of COVID-19 knowledge to be clear on what we do
and do not know. In this brief review, we enumerate ways that data
science has highlighted these weaknesses and is helping to address them.

In terms of understanding where we are, how we got here, and what is
likely to follow, here are some things we need to know. We need to know
the rate of spread of COVID-19 in a population \(R\), over time
\(R_{t}\), across different political and demographic communities
\(R_{ct}\), and prior to any non-pharmaceutical interventions
\(R_{c0}\). We need to know how many cases of active infection exist in
a community \(I_{ct}\) and how many of those infections resulted in
death \(D_{ct}\). We need to know the causal effect of interventions
\(X_{ct}\) on say rate of spread, between the observed treated
populations \(R_{ct1}\) and the counterfactual populations had they not
been treated \(R_{ct0}\). To do so, we need some plausible causal
identification strategy that allows us to account for the fact that
interventions are themselves chosen and implemented in response to
changes in \(R_{ct}\), and that many outside factors likely drive both
\(R_{ct}\) and \(X_{ct}\) simultaneously. These unknowns give rise to
fundamental problems of measurement, inference, and interpretation.

\hypertarget{measurement}{%
\section{Measurement}\label{measurement}}

There is no direct measure of \(I_{ct}\). The closest we have are
serological estimates of Cumulative Infections \(CI_{ct}\). These are
surveys that measure the presence of antibodies in a given population
which is an indication of what percent have had COVID-19 at some point
in the past, and cannot tell us how recently. They are still rarely
available and they have false positive rates that make them
inappropriate for populations early in their spread (Peeling et al.
2020). Further, they require careful sampling designs comparable to
opinion surveys, otherwise their findings may not be representative of
the full underlying population.

More commonly available are Confirmed Cases \(CC_{ct}\). COVID-19 tests
are administered in a jurisdiction, and positive results are anonymized,
tabulated, reported, and aggregated by increasingly nested
bureaucracies. These bureaucracies are concerned primarily with
releasing legally required contemporary measurements and not maintaining
consistent historical time series. This has resulted in the world's
largest, most desperate scavenger hunt to scrape, transcribe, and
translate counts disseminated in oral briefings, public websites, PDFs,
and even static images (Alamo et al. 2020). Teams from every country are
working in often uncoordinated and duplicated efforts to compile
government reporting into consistent panel data; these teams include
newspapers (Sun et al. 2020), nonprofits (USAFacts 2020), large private
companies (Zhang, Donthini, and Source 2020; Wolf, Ary, and Firooz
2020), consortiums of volunteers (Yang et al. 2020; Zohrab et al. 2020;
Group 2020), and Wikipedians.

The resulting ecosystem of panel datasets vary in spatial and temporal
coverage, have little metadata about sources or changing definitions,
and generally do not handle revisions to past counts from reporting
sources. Direct comparisons between sources reveal worrying
disagreements and temporal artifacts like reporting delays,
seasonalities, discontinuities, and sudden revisions in counts both
upwards and downwards (Wang et al. 2020). It is not obvious how to
correctly account for these problems or adjudicate between conflicting
sources without a clear ground truth. There also is no permanent archive
of the raw source material meaning reconstructing the full chain of
evidence may no longer be possible.

Likewise, we do not have direct measures of Deaths \(D_{ct}\) but only
Confirmed Deaths \(CD_{ct}\). \(CD_{ct}\) suffers from all of the
problems of Confirmed Cases \(CC_{ct}\) except for possibly less
under-reporting depending on if the person died at home or in medical
care. Choosing \(CD_{ct}\) as the lesser of two evils, many projects
attempt to take plausible values of the Infected Fatality Rate
\(IFR=D/I\) to back out an estimation for \(I_{ct}\) (Meyerowitz-Katz
and Merone 2020). Others have turned to estimating Excess Deaths
\(ED_{ct}\), which is a number proportional to the number of total
deaths reported in an area above what would be expected given the number
of deaths reported in previous years (Weinberger et al. 2020).
\(ED_{ct}\) is also not a direct estimate of \(D_{ct}\) as it can
include deaths that were not caused by COVID-19 directly, e.g.~other
health conditions that received inadequate care during this period, and
similarly can undercount the number of COVID-19 caused deaths as
lockdowns reduce mobility and economic activity that might typically
lead to deaths, e.g.~car accidents.

Confirmed case and death counts mechanically depend on testing, but
records of tests administered \(T_{ct}\) are even worse. In the U.S.,
much of what we know about trends in testing patterns come from
journalistic efforts like the Covid-Tracking Project (Lipton et al.
2020). They encountered all of the regular problems plus additional ones
specific to ambiguity to what kind of test count is being reported
(testing encounters, number of people tested, number of swabs tested,
etc). The type of test performed (and its false positive and false
negative rate) is almost never included as metadata. Nor are the rules
about how tests are being rationed and distributed being recorded
systematically.

The general failure to track COVID-19 spread directly has led to a
proliferation of innovative attempts to use other signals such as web
searches, searches of medical databases, social media posts, fevers
reported by home thermometer, and traditional flu symptom survailence
networks (Kogan et al. 2020). While promising, proxy measures require
ground truthing and regular calibration using something like regularly
timed serological surveys on smaller geographic samples of the
population. It is precisely the lack of such capabilities that are
motivating the search for alternatives in the first place.

Finally, non-pharmaceutical interventions are tracked by several
academic and nonprofit teams (Hale et al. 2020; Cheng et al. 2020).
These interventions are intended to limit human mobility which is more
directly measured by cell phone data which are being provided by
companies like Google, Apple, and SafeGraph.

\hypertarget{inference}{%
\section{Inference}\label{inference}}

The workhorse theoretical model for infectious disease spread is the
Susceptible, Exposed, Infectious, and Removed (SEIR) compartmental model
(Brauer and Castillo-Chavez 2012). The intuition behind the SEIR model
is that there are mechanical relationships, such as previous infections
or deaths removing candidates from infection, the timing between
exposure to the next possible transmission, and the degree to which
immunity may exist in the population, which induce nonlinearities in
disease spread. Disease spreads slowly at first, accelerates, and then
burns out if left to its own devices. SEIR should be considered the
theoretical floor for analysis, and a entire menagerie of extensions
account for demography, testing, mobility, social networks, etc.

The necessity of directly including testing in models of disease spread
can't be understated. Per capita cases are so temporarily correlated
with per capita testing rates they are more of a proxy of testing
availability than actual cases (Kaashoek and Santillana 2020).
Spatially, per capita testing rates correlate with urbanity and a wide
range of co-morbids (Souch and Cossman 2020). How many tests are given
and to who varies systematically in response to conditions on the ground
with both periods of rationing and blitzes.

Measuring the effect of interventions is difficult because they are
assigned endogenously in response to both local conditions and national
signals. Similarly, populations responded to both government orders and
local conditions, often reducing their activity prior to being ordered
to and also increasing their activity prior to being officially allowed
to. Governments, the public, and the disease are all responding
simultaneously to each other in often nonlinear and unobserved ways.
Statistical instruments that cause government interventions but do not
directly cause testing rates or rate of spread except through the
government intervention are few and far between. Further, interventions
are often implemented simultaneously or in a rolling cumulative pattern
directly in response to changes in cases and testing results, making
isolating the effect of any one treatment exceptionally difficult.

Even if we had an exogenous intervention, its treatment effect on the
rate of spread is still unlikely to be identified since almost any
intervention will affect both cases and testing. Estimating an effect on
just spread requires imposing additional assumptions, e.g.~sharp
constraints on some parameters and informative priors on the
relationship between the number of tests and the number of cases
(Kubinec and Carvalho 2020).

\hypertarget{interpretation}{%
\section{Interpretation}\label{interpretation}}

One promising development is rigorous forecast evaluations (Reich et al.
2020). Notoriously, many early simple growth models fit to the takeoff
period of infections performed well right up until the curve broke and
then failed entirely. A parade of predicted peaks in cases since
continue the tradition, with groups celebrating success on uninteresting
short-term auto-correlations while ignoring failures on actually
interesting shifts in trends. All we can do is develop a very long
memory of predictions and constantly hold models accountable for their
long run out-of-sample performance on unseen future data.

Other trends in COVID-19 work are less promising.\footnote{We omit
  citations falling under the criticisms provided below as they are
  working papers and likely to change before finalization.} An
overabundance of observational work still presents correlations as
evidence of causation. Without identification, correlations on short
highly autocorrelated time series are as likely to be misleading as
informative. The SEIR model expects a nonlinear and highly
auto-correlated pattern of an increasing infection rate that then levels
off independent of any interventions. An unscrupulous, or naive, analyst
can easily find interventions that increased (or decreased) spread
solely by where those interventions land in the natural disease cycle,
completely independent of the intervention's actual effect.

Another bad habit is the pursuit of statistically distinguishable
correlations over actually attempting to explain variation in COVID-19
outcomes themselves. Papers that can show a particular political party
or demographic group is `worse' on some COVID-19 dimension receive much
attention. Such results lack strong explanatory power or clear policy
recommendations, and so while great for making headlines, they do little
to help us end the current pandemic.

Perhaps the most egregious trend in recent scholarship is setting up
straw man null hypothesis and then presenting the inability to reject
them as positive evidence for medical and safety decisions, e.g.~social
distancing might not be required because a model was unable to
statistically distinguish a large uptick in cases following a
mass-meeting. In the best of circumstances, absence of evidence is not
evidence of absence. Our underfit, undertheorized, and underperforming
observational models are not the best of circumstances, and they are not
sufficiently sensitive to evaluate more than macro-level general trends.

\hypertarget{conclusion}{%
\section{Conclusion}\label{conclusion}}

This necessarily brief review omitted positive developments in studying
COVID-19 outside of macro-observational settings. There has been
remarkable progress in areas of diagnosis, clinical treatment, and
phylogenetic tracking. Data science has contributed to the rapid
collaboration, development, and dissemination of research in a way not
seen in prior disease outbreaks. We also neglected topics like tracing,
and the accompanying contributions from the tech field such as
monitoring through mobile apps and social media. Further, our review is
overly U.S.-centric, with other countries like South Korea monitoring
the disease so effectively they succeeded at containment without having
to resort to difficult mitigation.

Any research related to COVID-19 requires healthy caution of and respect
for how little we actually know about the history of this pandemic.
Practioners working on these questions and with these data will be
deeply familiar with many of these concerns, but some may be especially
subtle or less prominant within one's main field of study. At a minimum,
there is research being produced today which ignores much of these known
methodological problems and subsequently generates confusion for novice
consumers of analysis. We hope this enumeration of challenges in
measurement, inference, and interpretation, can help both consumers and
producers of COVID-19 knowledge alike.

\hypertarget{references}{%
\section{References}\label{references}}

\hypertarget{refs}{}
\begin{cslreferences}
\leavevmode\hypertarget{ref-alamoCovid19OpenDataResources2020}{}%
Alamo, Teodoro, Daniel G. Reina, Martina Mammarella, and Alberto Abella.
2020. ``Covid-19: Open-Data Resources for Monitoring, Modeling, and
Forecasting the Epidemic.'' \emph{Electronics} 9 (5): 827.
\url{https://doi.org/10.3390/electronics9050827}.

\leavevmode\hypertarget{ref-brauerMathematicalModelsPopulation2012}{}%
Brauer, Fred, and Carlos Castillo-Chavez. 2012. \emph{Mathematical
Models in Population Biology and Epidemiology}. Vol. 2. Springer.

\leavevmode\hypertarget{ref-chengCOVID19GovernmentResponse2020a}{}%
Cheng, Cindy, Joan Barceló, Allison Spencer Hartnett, Robert Kubinec,
and Luca Messerschmidt. 2020. ``COVID-19 Government Response Event
Dataset (CoronaNet V.1.0).'' \emph{Nature Human Behaviour} 4 (7):
756--68. \url{https://doi.org/10.1038/s41562-020-0909-7}.

\leavevmode\hypertarget{ref-covid19indiaorg2020tracker}{}%
Group, COVID-19 India Org Data Operations. 2020. ``Dataset for Tracking
COVID-19 Spread in India.''

\leavevmode\hypertarget{ref-haleVariationGovernmentResponses2020}{}%
Hale, Thomas, Anna Petherick, Toby Phillips, and Samuel Webster. 2020.
``Variation in Government Responses to COVID-19.'' \emph{Blavatnik
School of Government Working Paper} 31.

\leavevmode\hypertarget{ref-kaashoekCOVID19PositiveCases2020a}{}%
Kaashoek, Justin, and Mauricio Santillana. 2020. ``COVID-19 Positive
Cases, Evidence on the Time Evolution of the Epidemic or an Indicator of
Local Testing Capabilities? A Case Study in the United States.'' SSRN
Scholarly Paper ID 3574849. Rochester, NY: Social Science Research
Network. \url{https://doi.org/10.2139/ssrn.3574849}.

\leavevmode\hypertarget{ref-koganEarlyWarningApproach2020}{}%
Kogan, Nicole E., Leonardo Clemente, Parker Liautaud, Justin Kaashoek,
Nicholas B. Link, Andre T. Nguyen, Fred S. Lu, et al. 2020. ``An Early
Warning Approach to Monitor COVID-19 Activity with Multiple Digital
Traces in Near Real-Time.'' \emph{arXiv:2007.00756 {[}Q-Bio, Stat{]}},
July. \url{http://arxiv.org/abs/2007.00756}.

\leavevmode\hypertarget{ref-kubinecRetrospectiveBayesianModel2020}{}%
Kubinec, Robert, and Luiz Carvalho. 2020. ``A Retrospective Bayesian
Model for Measuring Covariate Effects on Observed COVID-19 Test and Case
Counts,'' April. \url{https://doi.org/10.31235/osf.io/jp4wk}.

\leavevmode\hypertarget{ref-zachliptonCovidTrackingProject2020}{}%
Lipton, Zach, Josh Ellington, smike, James Ouyang, Ken Riley, Joshua
Ellinger, Jeff Hammerbacher, Olivier Lacan, Jason Crane, and
space-buzzer. 2020. ``The Covid-Tracking Project.'' Zenodo.
\url{https://doi.org/10.5281/zenodo.3981599}.

\leavevmode\hypertarget{ref-meyerowitz-katzSystematicReviewMetaanalysis2020}{}%
Meyerowitz-Katz, Gideon, and Lea Merone. 2020. ``A Systematic Review and
Meta-Analysis of Published Research Data on COVID-19 Infection-Fatality
Rates.'' \emph{medRxiv}, May, 2020.05.03.20089854.
\url{https://doi.org/10.1101/2020.05.03.20089854}.

\leavevmode\hypertarget{ref-peelingSerologyTestingCOVID192020}{}%
Peeling, Rosanna W., Catherine J. Wedderburn, Patricia J. Garcia, Debrah
Boeras, Noah Fongwen, John Nkengasong, Amadou Sall, Amilcar Tanuri, and
David L. Heymann. 2020. ``Serology Testing in the COVID-19 Pandemic
Response.'' \emph{The Lancet Infectious Diseases} 0 (0).
\url{https://doi.org/10.1016/S1473-3099(20)30517-X}.

\leavevmode\hypertarget{ref-nicholasgreichReichlabCovid19forecasthubPrepublication2020}{}%
Reich, Nicholas G, Jarad Niemi, Katie House, Abdul Hannan, Estee Cramer,
Steve Horstman, Shanghong Xie, et al. 2020.
``Reichlab/Covid19-Forecast-Hub: Pre-Publication Snapshot.'' Zenodo.
\url{https://doi.org/10.5281/zenodo.3963372}.

\leavevmode\hypertarget{ref-souchCommentaryRuralUrban2020}{}%
Souch, Jacob M., and Jeralynn S. Cossman. 2020. ``A Commentary on
Rural-Urban Disparities in COVID-19 Testing Rates Per 100,000 and Risk
Factors.'' \emph{The Journal of Rural Health}, June, jrh.12450.
\url{https://doi.org/10.1111/jrh.12450}.

\leavevmode\hypertarget{ref-albertsunNewYorkTimes2020}{}%
Sun, Albert, Tiff Fehr, Archie Tse, Rachel, and Wilson Andrews. 2020.
``New York Times Coronavirus (Covid-19) Data in the United States.''
Zenodo. \url{https://doi.org/10.5281/zenodo.3981451}.

\leavevmode\hypertarget{ref-usafactsUSCoronavirusCases2020}{}%
USAFacts. 2020. ``US Coronavirus Cases and Deaths.'' Zenodo.
\url{https://doi.org/10.5281/zenodo.3981486}.

\leavevmode\hypertarget{ref-wangComparingIntegratingUS2020}{}%
Wang, Guannan, Zhiling Gu, Xinyi Li, Shan Yu, Myungjin Kim, Yueying
Wang, Lei Gao, and Li Wang. 2020. ``Comparing and Integrating US
COVID-19 Daily Data from Multiple Sources: A County-Level Dataset with
Local Characteristics.'' \emph{arXiv:2006.01333 {[}Stat{]}}, June.
\url{http://arxiv.org/abs/2006.01333}.

\leavevmode\hypertarget{ref-weinbergerEstimationExcessDeaths2020}{}%
Weinberger, Daniel M., Jenny Chen, Ted Cohen, Forrest W. Crawford,
Farzad Mostashari, Don Olson, Virginia E. Pitzer, et al. 2020.
``Estimation of Excess Deaths Associated with the COVID-19 Pandemic in
the United States, March to May 2020.'' \emph{JAMA Internal Medicine},
July. \url{https://doi.org/10.1001/jamainternmed.2020.3391}.

\leavevmode\hypertarget{ref-ashleywolfYahooKnowledgeGraph2020}{}%
Wolf, Ashley, Asaf Ary, and Hossein Firooz. 2020. ``Yahoo Knowledge
Graph COVID-19 Datasets.'' Zenodo.
\url{https://doi.org/10.5281/zenodo.3981432}.

\leavevmode\hypertarget{ref-yangCovidNetBringData2020}{}%
Yang, Tong, Kai Shen, Sixuan He, Enyu Li, Peter Sun, Pingying Chen, Lin
Zuo, et al. 2020. ``CovidNet: To Bring Data Transparency in the Era of
COVID-19.'' \emph{arXiv:2005.10948 {[}Cs, Q-Bio{]}}, July.
\url{http://arxiv.org/abs/2005.10948}.

\leavevmode\hypertarget{ref-chiqunzhangBingCOVID19Data2020}{}%
Zhang, Chiqun, Chaitanya Donthini, and Microsoft Open Source. 2020.
``Bing-COVID-19-Data.'' Zenodo.
\url{https://doi.org/10.5281/zenodo.3978733}.

\leavevmode\hypertarget{ref-jzohrabCOVIDAtlasLi2020}{}%
Zohrab, J, Ryan Block, Cameron Chamberlain, Larry Davis, Minh Nguyeñ̂,
Alastair Gifillan, Adam Hughes, BriceWolfgang, and andys1376. 2020.
``COVID Atlas Li.'' Zenodo.
\url{https://doi.org/10.5281/zenodo.3981563}.
\end{cslreferences}

\end{document}
